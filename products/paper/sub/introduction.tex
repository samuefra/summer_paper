Within the predictive world of finance, financial economists try to find methods and models that can increasingly more accurately predict stock returns. There are several approaches and tools used in financial analysis. These techniques are used to analyze historical data and try to predict future stock market movements.

On the one hand, the literature walked for a long time trying to discover financial factors that could better explain the returns or that would increase the predictive power of models that already have significant factors when making forecasts of returns. The search for factors that explain the cross section of expected stock returns began a long time ago with its first model, the Capital Asset Pricing Model (CAPM) introduced by \citeonline{sharpe1964capital}, \citeonline{lintner1965security} and \citeonline{mossin1966equilibrium} independently , based on previous work by \citeonline{markowitz1952portfolio}, using only the market factor to explain stocks' excess returns.

The factors are formed through portfolios where there is a long and a short position. For example, to create the market factor, we use a long position on the market index of interest and short on the risk-free interest rate, that is, market return minus risk-free asset return.

Looking for more factors that increase the predictive power, the well-known Fama and French models emerged, respectively, the three-factor model in \citeonline{fama1992cross} and the five-factor model in \citeonline{fama2015five}.

The first model aggregates the factors related to size, SMB (Small minus Big), and value, HML (High minus Low), based respectively on the variables Market Capitalization and Book-to-Market ratio, keeping the market factor at your model. In the second model, the authors add two more factors related to profitability and investment. Defined analogously to the SMB and HML factors, the RMW (Robust minus Weak) factors are constructed from the difference between the returns of firms with robust (high) and weak (low) operating profitability and the CMA investment factor (Conservative minus Aggressive) constructed from the difference between the returns of companies that invest conservatively and companies that invest aggressively. Another factor related to momentum was also added to the Fama and French three-factor model by \citeonline{carhart1997persistence}. MOM (Monthly Momentum Factor) can be calculated by subtracting the equal weighted average of the lowest performing companies from the equal weighted average of the highest performing companies.

Now, with hundreds of financial factors released in the literature, the difficulty has become to tame this high dimensionality of factors.

On the other hand, there is still a great use of traditional methods, such as the use of autoregressive regression (AR). This is what is done in \citeonline{chinco2019sparse}, which is the article most closely linked to this work. In the article, the authors use the AR regression with three lags as a reference model in their main specifications, although they demonstrate that the number of lags does not matter with the problem addressed. The objective of the article is to show that the identification of predictors in finance using only the intuition of researchers would only work for long-term stable predictors.

In high-frequency finance with modern, large, fast, and complex financial markets, the process of selecting predictors requires efforts that go beyond researchers' intuition. And perhaps these long-term stable predictors are not suitable to explain intraday returns because they are not able to capture the effects that affect stock prices in the middle of a day. The paper's results show that statistical model selection tools can increase the predictive power of one-minute-ahead forecasts of benchmark models using out-of-sample fit and forecast-implied Sharpe ratios as measures of quality.

For model selection, the authors make use of LASSO (Least Absolute Shrinkage and Selection Operator) using three lags of all returns as candidate predictors. They ague the use of a dimension reduction tool because there are many predictor candidates. For example, in the month of January 2003 our sample contains $4500+$ stocks. By using three lags, we then have $3\cdot 4500+ = 13500+$ candidate predictors. It would take more than 34 trading days to do a simple one-minute OLS (Ordinary Least Squares) estimation (each day has 390 minutes/observations) and test the predictors.

With this high dimensionality of predictor candidates, a reduction tool is necessary and a researcher's intuition does not seem to be the right path for this scale. LASSO fits as a solution because it is able to identify unexpected and short-term predictors, suitable for an intraday financial predictive model. The initial hypothesis for using LASSO is to bet on sparsity, that is, if among the $13500+$ candidate predictors, only a few predictors, say $S$, actually help to predict the returns of an asset, then we can leverage to use LASSO, because it would be needed only a little more than $S$ observations to make predictions, this means that, since we don't have to worry about the weak estimators, LASSO can estimate the remaining parameters with far fewer observations. Thus, if there are only $S$ important predictors at each point in time, LASSO is adequate for estimating unexpected short-term parameters. We explain more about how LASSO works in Appendix \ref{apen:lasso}.

Although the paper obtains results that refer to gains in predictive power by combining LASSO with Benchmarks models, the selection of predictors is made based only on return lags. The factors literature is neglected and therefore, this work was motivated to verify what happens when we add to this predictive model a base that contains relevant financial factors.

This summer paper therefore aims to explain how the database will be created and report the next steps after that. The work is divided into two more sections. Section \ref{sec:data} addresses how the databases have been built so far and how theoretically the factors will be built. Section \ref{sec:conclusion} will motivate what the next steps will be after having the base ready.




%% PORTUGUESE VERSION

%Dentro do mundo preditivo de finanças, os economistas financeiros tentam encontrar métodos e modelos que possam cada vez mais, fazer uma previsão mais precisa dos retornos das ações. Existem diversas abordagens e ferramentas utilizadas na análise financeira. Essas técnicas são usadas para analisar dados históricos e tentar prever futuros movimentos do mercado de ações.
%
%Por um lado, a literatura caminhou por muito tempo tentando descobrir fatores financeiros que pudessem explicar melhor os retornos ou que aumentassem o poder preditivo de modelos que já possuem fatores significativos na hora de fazer previsões de retornos. A procura por fatores que expliquem o cross section de retornos esperados de ações começou há muito tempo com seu primeiro modelo, Capital Asset Pricing Model (CAPM) introduzido por \citeonline{sharpe1964capital}, \citeonline{lintner1965security} e \citeonline{mossin1966equilibrium} independentemente, com base no trabalho anterior de \citeonline{markowitz1952portfolio}, usando apenas o fator mercado para explicar o excesso de retorno das ações. 
%
%Os fatores são formados através de portfólios onde há uma posição comprada e uma vendida. Por exemplo, para se criar o fator de mercado, usamos uma posição comprada no índice de mercado de interesse e vendido na taxa de juros livre de risco, isso é, retorno de mercado menos retorno do ativo livre de risco. 
%
%Procurando por mais fatores que aumentem o poder preditivo, surgiram os conhecidos modelos de Fama e French, respectivamente, modelo de três fatores em \citeonline{fama1992cross} e modelo de cinco fatores em \citeonline{fama2015five}. 
%
%O primeiro modelo agrega os fatores relacionados ao tamanho, SMB (Small minus Big), e valor, HML (High minus Low), baseando-se respectivamente, nas variáveis Market Capitalization e razão Book-to-Market, mantendo o fator de mercado em seu modelo. No segundo modelo, os autores acrescentam ainda mais dois fatores relacionados à lucratividade e investimento. Definidos de forma análoga aos fatores SMB e HML, são os fatores RMW (Robust minus Weak) construído a partir da diferença entre os retornos das firmas com rentabilidade operacional robusta (alta) e fraca (baixa) e o fator de investimento CMA (Conservative minus Aggressive) construído a partir da diferença entre os retornos das empresas que investem de forma conservadora e das empresas que investem de forma agressiva. Outro fator relacionado ao momentum também foi adicionado ao modelo trifatorial de Fama e French por \citeonline{carhart1997persistence}. O MOM (Monthly Momentum Factor) pode ser calculado subtraindo-se a média ponderada igual das empresas com desempenho mais baixo a partir da média ponderada igual das empresas com desempenho mais alto. 
%
%Agora, com centenas de fatores financeiros lançados na literatura, a dificuldade passou a ser domar essa alta dimensionalidade de fatores.
%
%Por outro lado, ainda existe uma grande utilização de métodos tradicionais, como o uso de regressão autorregressiva (AR). Isso é o que é feito em \citeonline{chinco2019sparse}, que é o artigo mais intimamente ligado a este trabalho. No artigo, os autores utilizam a regressão AR com três defasagens como modelo de referência em suas principais especificações, embora demonstrem que o número de defasagens não importa com o problema abordado. O objetivo do artigo é mostrar que a identificação de preditores em finanças utilizando somente a intuição de pesquisadores só funcionaria para preditores estáveis de longa duração. 
%
%Em finanças de alta frequência com mercados financeiros modernos, grandes, rápidos e complexos, o processo de seleção de preditores exige esforços que vão além da intuição de pesquisadores. E talvez esses preditores estáveis de longa duração não sejam adequados para explicar retornos intra diários por não serem capaz de capturar os efeitos que afetam os preços de ações no meio de um dia. Os resultados do paper mostram que ferramentas estatísticas de seleção de modelo podem aumentar o poder preditivo de previsões de um minuto à frente de modelos benchmark usando out-of-sample fit e forecast-implied Sharpe ratios como medidas de qualidade. 
%
%Para a seleção de modelo, os autores fazem uso do LASSO (Least Absolute Shrinkage and Selection Operator) usando três defasagens de todos os retornos como preditores candidatos. Eles defendem o uso de uma ferramenta de redução de dimensão por existirem muitos candidatos à preditores. Por exemplo, no mês de Janeiro de 2003 nossa amostra contém $4500+$ ações. Ao utilizar três defasagens, temos então $3\cdot 4500+ = 13500+$ candidatos a preditores. Seria necessário mais de 34 dias de negociações para fazer uma simples estimação OLS na frequência de um minuto (Ordinary Least Squares) (cada dia tem 390 minutos/observações) e testar os preditores. 
%
%Com essa alta dimensionalidade de candidatos a preditores, uma ferramenta de redução é necessária e a intuição de um pesquisador parece não ser o caminho adequado para essa escala. O LASSO se encaixa como solução por ser capaz de identificar preditores inesperados e de curta-duração, adequado para um modelo preditivo financeiro intra diário. A hipótese inicial para se usar o LASSO é apostar em sparsity, isso é, se dentre o $13500+$ candidatos a preditores, apenas alguns poucos preditores, digamos $S$, de fato ajudam a prever os retornos de um ativo, então podemos aproveitar para usar o LASSO, pois seria necessário apenas pouco mais de $S$ observações para fazer previsões, isso quer dizer que, como não precisamos nos preocupar com os estimadores fracos, o LASSO pode estimar os parâmetros remanescentes com muito menos observações. Assim, se existem apenas $S$ preditores importantes em cada ponto do tempo, o LASSO é adequado na estimação de parâmetros inesperados de curta-duração. Explicamos melhor como o LASSO funciona no Apêndice \ref{apen:lasso}.
%
%Embora o paper obtenha resultados que referem-se a ganhos no poder preditivo aliando o LASSO à modelos Benchmarks, a seleção de preditores é feita sobre apenas defasagens de retornos. A literatura de fatores é negligenciada e por isso, esse trabalho motivou-se em verificar o que acontece quando incrementamos à esse modelo preditivo uma base que contenha fatores financeiros relevantes.
%
%Esse artigo de verão tem como objetivo portanto, explicar como a base de dados será criada e reportar os próximos passos após isso. O trabalho está dividido em mais duas seções. A Seção \ref{sec:data} aborda como as bases de dados foram construídas até agora e como teoricamente, os fatores serão construídos. A Seção \ref{sec:conclusion} motivará quais serão os próximos passos após ter a base pronta.